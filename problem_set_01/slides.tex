\documentclass{beamer}

\usetheme{Madrid}

\title{Designing Markets}
\subtitle{Chapter 4 — Handbook of Market Design}
\author{Your Name \and Partner Name}
\institute{ECN 591 — Market Design}
\date{\today}

\begin{document}

% Slide 1
\begin{frame}
  \titlepage
\end{frame}

% Slide 2
\begin{frame}{What Is Market Design?}
\begin{itemize}
  \item Market design studies how trading rules affect outcomes.
  \item The goal is to create mechanisms that lead to efficient, fair, and stable outcomes.
  \item Examples include matching markets, auctions, and centralized allocation systems.
\end{itemize}
\end{frame}

% Slide 3
\begin{frame}{Objectives of Market Design}
\begin{itemize}
  \item Efficiency: allocate resources to maximize total surplus.
  \item Incentives: encourage truthful reporting of preferences.
  \item Stability: prevent participants from bypassing the mechanism.
  \item Fairness: ensure acceptable outcomes for participants.
\end{itemize}
\end{frame}

% Slide 4
\begin{frame}{Diagnosing Market Failures}
\begin{itemize}
  \item Congestion and coordination failures.
  \item Strategic manipulation and misreporting.
  \item Lack of thickness (too few participants).
  \item Timing problems and unraveling.
\end{itemize}
\end{frame}

% Slide 5
\begin{frame}{Examples of Market Failures}
\begin{itemize}
  \item School choice systems with unstable matchings.
  \item Kidney exchange markets without centralized matching.
  \item Labor markets with early contracting.
\end{itemize}
\end{frame}

% Slide 6
\begin{frame}{Design Tools}
\begin{itemize}
  \item Matching algorithms (e.g., deferred acceptance).
  \item Auctions and pricing mechanisms.
  \item Centralization of information and timing.
\end{itemize}
\end{frame}

% Slide 7
\begin{frame}{Evaluating Market Designs}
\begin{itemize}
  \item Does the mechanism achieve efficiency?
  \item Are incentives aligned with truthful behavior?
  \item Is the outcome stable?
  \item Are participants willing to use the system?
\end{itemize}
\end{frame}

% Slide 8
\begin{frame}{Comparing Alternative Designs}
\begin{itemize}
  \item Trade-offs between efficiency and fairness.
  \item Simplicity versus optimality.
  \item Robustness to strategic behavior.
\end{itemize}
\end{frame}

% Slide 9
\begin{frame}{Policy Implications}
\begin{itemize}
  \item Well-designed markets can correct failures without heavy regulation.
  \item Small design choices can have large effects.
  \item Empirical evaluation is crucial.
\end{itemize}
\end{frame}

% Slide 10
\begin{frame}{Conclusion}
\begin{itemize}
  \item Market design provides tools to improve real-world allocation problems.
  \item Proper diagnosis is essential before proposing solutions.
  \item Chapter 4 highlights principles used across many applications.
\end{itemize}
\end{frame}

\end{document}
